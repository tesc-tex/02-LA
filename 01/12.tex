\subsection{%
  Билинейные формы: определения, свойства. Матрица билинейной формы.%
}

\begin{definition}
  Функция \(\BiLinear \colon V^{n} \to \RR\) обозначаемая
  \(\BiLinear(u, v) \; (u, v \in V^{n})\) называется билинейной формой, если
  выполняются следующие требования:
  \(\forall u, w, v \in V^{n}, \lambda \in \RR\):

  \begin{enumerate}
    \item \(\BiLinear(u + w, v) = \BiLinear(u, v) + \BiLinear(w, v)\)
    \item \(\BiLinear(u, v + w) = \BiLinear(u, v) + \BiLinear(u, w)\)
    \item \(\BiLinear(\lambda u, v) = \lambda \BiLinear(u, v)\)
    \item \(\BiLinear(u, \lambda v) = \lambda \BiLinear(u, v)\)
  \end{enumerate}
\end{definition}

\begin{definition}
  Если к каждой паре базисных векторов применить билинейную форму, то
  полученные числа можно использовать как коэффициенты матрицы. Это матрица
  будет называться матрицей билинейной формы \textbf{в данном базисе}.

  Таким образом матрица \(B\) билинейной формы \(\BiLinear\) в базисе
  \(\Basis = \{ \basis \}_{i = 1}^{n}\) имеет вид:

  \begin{align*}
    \begin{pmatrix}
      \BiLinear(\basis_{1}, \basis_{1})
        & \dots & \BiLinear(\basis_{1}, \basis_{n}) \\
      \vdots & \ddots & \vdots \\
      \BiLinear(\basis_{n}, \basis_{1})
        & \dots & \BiLinear(\basis_{n}, \basis_{n})
    \end{pmatrix}
  \end{align*}
\end{definition}

\begin{definition}
  Билинейная форма \(\BiLinear\) называется симметричной, если
  \(\BiLinear(u, v) = \BiLinear(v, u)\).
\end{definition}

\begin{definition}
  Билинейная форма \(\BiLinear\) называется кососимметричной (антисимметричной),
  если \(\BiLinear(u, v) = -\BiLinear(v, u)\).
\end{definition}

\begin{remark}
  Применение билинейной формы \(\BiLinear\) к элементам \(u\) и \(v\) можно
  отождествлять с умножением матриц в виде \(u^{T} B v\).

  Тогда можно говорить о ранге билинейной формы и о её преобразовании при смене
  базиса.
\end{remark}

\begin{lemma}
  Ранг билинейной формы это инвариант относительно смены базиса \(T\).
\end{lemma}
\begin{proof}
  \(B_{\basis'} = T_{\basis' \to \basis} B_{\basis} T_{\basis \to \basis'}\)

  Т.к. матрица \(T_{\basis' \to \basis}\) невырождена, то
  \(\Rang B_{\basis'} = \Rang B_{\basis}\)
\end{proof}

\begin{definition}
  Если ранг билинейной формы \(\BiLinear \colon V^{n} \to \RR\) равен \(n\), то
  такая билинейная форма называется невырожденной.
\end{definition}
