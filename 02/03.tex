\subsection{%
  Однородное уравнение.%
}

\begin{definition}
  Функция \(f(x, y)\) называется \textit{однородной} \(m\)-ого измерения
  (\(m \ge 0\)), если \(f(\lambda x, \lambda y) = \lambda^{m} f(x, y)\).
\end{definition}

\begin{definition}
  Дифференциальное уравнение \(P(x, y) \dd x + Q(x, y) \dd y = 0\) называется
  \textit{однородным}, если \(P(x, y)\) и \(Q(x, y)\) однородные функции 
  одного измерения \(m\).
\end{definition}

Однородные уравнения решаются заменой \(t = \dfrac{y}{x}\). Покажем, откуда
появляется подобная замена. Преобразуем функции \(P(x, y)\) и \(Q(x, y)\):

\begin{align*}
  P(x, y) = P \left( x \cdot 1, x \cdot \frac{y}{x} \right)
    = x^{m} P \left( 1, \frac{y}{x} \right) \\
  Q(x, y) = Q \left( x \cdot 1, x \cdot \frac{y}{x} \right)
    = x^{m} Q \left( 1, \frac{y}{x} \right)
\end{align*}

Вернемся к исходному уравнению:

\begin{align*}
  P(x, y) \dd x + Q(x, y) \dd y = 0 \mid \colon \dd x \\
  y'
  = -\frac{P(1, \sfrac{y}{x})}{Q(1, \sfrac{y}{x})}
  = f \left( 1, \frac{y}{x} \right) \\
  \frac{y}{x} = t \implies \begin{cases}
    f(1, \sfrac{y}{x}) = \tilde{f}(t) \\
    y = xt, \; y'_{x} = t + x t'
  \end{cases} \\
  t + x t' = \tilde{f}(t) \\
  x \cdot \frac{\dd t}{\dd x} = \tilde{f}(t) - t \\
  \frac{\dd t}{\tilde{f}(t) - t} = \frac{\dd x}{x}
\end{align*}

Таким образом исходное однородное уравнение сводится к уравнению с разделяющими
переменными.

\begin{remark}
  Случай \(\tilde{f}(t) - t = 0\) нужно рассмотреть отдельно. Также в ходе
  решения было произведено деление на \(\dd x \implies\) нужно рассмотреть
  случай \(x = const\).
\end{remark}
